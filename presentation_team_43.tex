%%%%%%%%%%%%%%%%%%%%%%%%%%%%%%%%%%%%%%%%%%%%%%%%%%
\documentclass{beamer}

\mode<presentation> {

\usepackage{pgfgantt}
\usetheme{Madrid}
}

\usepackage{graphicx} 
\usepackage{booktabs} 

\title[Short Title]{Title of the Project }
\logo{\includegraphics[width=.8cm,keepaspectratio]{logo.png}}
%\author[Pooja Agarwal]
\author[]{Mohinish Nandwana
\newline
Riyan Dsouza
\newline
Anmol Pandita
\newline	
Spandana Konduru} 
\institute[Batch]{Guide's Name - Mrs. Kirti Torvi
\newline
Batch No. - 43}
\date{\today} 

\begin{document}

\begin{frame}
\titlepage 
\end{frame}

\begin{frame}
\frametitle{Problem Statement / Definition} 
\tableofcontents % Throughout your presentation, if you choose to use \section{} and \subsection{} commands, these will automatically be printed on this slide as an overview of your presentation
\end{frame}

%----------------------------------------------------------------------------------------
%	PRESENTATION SLIDES
%----------------------------------------------------------------------------------------

%------------------------------------------------
\section{First Section} 

\subsection{Details} % A subsection can be created just before a set of slides with a common theme to further break down your presentation into chunks

\begin{frame}
\frametitle{Motivation of the Work}
Details

\end{frame}

%------------------------------------------------

\begin{frame}
\frametitle{Literature Survey}
\begin{itemize}
\item Andrew et al explained......
\item XYZ, ABC explained.............
\end{itemize}
\end{frame}

%------------------------------------------------
\begin{frame}
\frametitle{Methodology}
Details
\end{frame}



%------------------------------------------------

\begin{frame}
\frametitle{Time line of completion of project from Nov 2018-10th April 2019(Gantt Charts).}
\definecolor{barblue}{RGB}{153,204,254}
\definecolor{groupblue}{RGB}{51,102,254}
\definecolor{linkred}{RGB}{165,0,33}
\renewcommand\sfdefault{phv}
\renewcommand\mddefault{mc}
\renewcommand\bfdefault{bc}
\setganttlinklabel{s-s}{START-TO-START}
\setganttlinklabel{f-s}{FINISH-TO-START}
\setganttlinklabel{f-f}{FINISH-TO-FINISH}
\sffamily
\begin{ganttchart}[
    canvas/.append style={fill=none, draw=black!5, line width=.75pt},
    hgrid style/.style={draw=black!5, line width=.75pt},
    vgrid={*1{draw=black!5, line width=.75pt}},
    today=7,
    today rule/.style={
      draw=black!64,
      dash pattern=on 3.5pt off 4.5pt,
      line width=1.5pt
    },
    today label font=\small\bfseries,
    title/.style={draw=none, fill=none},
    title label font=\bfseries\footnotesize,
    title label node/.append style={below=7pt},
    include title in canvas=false,
    bar label font=\mdseries\small\color{black!70},
    bar label node/.append style={left=2cm},
    bar/.append style={draw=none, fill=black!63},
    bar incomplete/.append style={fill=barblue},
    bar progress label font=\mdseries\footnotesize\color{black!70},
    group incomplete/.append style={fill=groupblue},
    group left shift=0,
    group right shift=0,
    group height=.5,
    group peaks tip position=0,
    group label node/.append style={left=.6cm},
    group progress label font=\bfseries\small,
    link/.style={-latex, line width=1.5pt, linkred},
    link label font=\scriptsize\bfseries,
    link label node/.append style={below left=-2pt and 0pt}
  ]{1}{13}
  \gantttitle[
    title label node/.append style={below left=7pt and -3pt}
  ]{WEEKS:\quad1}{1}
  \gantttitlelist{2,...,13}{1} \\
  \ganttgroup[progress=57]{WBS 1 Summary Element 1}{1}{10} \\
  \ganttbar[
    progress=75,
    name=WBS1A
  ]{\textbf{WBS 1.1} Activity A}{1}{8} \\
  \ganttbar[
    progress=67,
    name=WBS1B
  ]{\textbf{WBS 1.2} Activity B}{1}{3} \\
  \ganttbar[
    progress=50,
    name=WBS1C
  ]{\textbf{WBS 1.3} Activity C}{4}{10} \\
  \ganttbar[
    progress=0,
    name=WBS1D
  ]{\textbf{WBS 1.4} Activity D}{4}{10} \\[grid]
  
\end{ganttchart}

%\begin{columns}[c] % The "c" option specifies centered vertical alignment while the "t" option is used for top vertical alignment

%\column{.45\textwidth} % Left column and width
%\textbf{Heading}
%\begin{enumerate}
%\item Statement
%\item Explanation
%\item Example
%\end{enumerate}

%\column{.5\textwidth} % Right column and width
%Lorem ipsum dolor sit amet, consectetur adipiscing elit. Integer lectus nisl, ultricies in feugiat rutrum, porttitor sit amet augue. Aliquam ut tortor mauris. Sed volutpat ante purus, quis accumsan dolor.

%\end{columns}
\end{frame}
\begin{frame}
\frametitle{Expected Outcome/ Results}
\begin{block}{1.}
Details
\end{block}

\begin{block}{2.}
Details
\end{block}

\begin{block}{3.}
Details
\end{block}
\end{frame}
%------------------------------------------------
%\section{Second Section}
%------------------------------------------------

%\begin{frame}
%\frametitle{Table}
%\begin{table}
%\begin{tabular}{l l l}
%\toprule
%\textbf{Treatments} & \textbf{Response 1} & \textbf{Response 2}\\
%\midrule
%Treatment 1 & 0.0003262 & 0.562 \\
%Treatment 2 & 0.0015681 & 0.910 \\
%Treatment 3 & 0.0009271 & 0.296 \\
%\bottomrule
%\end{tabular}
%\caption{Table caption}
%\end{table}
%\end{frame}

%------------------------------------------------



%------------------------------------------------

%\begin{frame}
%\frametitle{Figure}
%Uncomment the code on this slide to include your own image from the same directory as the template .TeX file.
%\begin{figure}
%\includegraphics[width=0.8\linewidth]{test}
%\end{figure}
%\end{frame}

\begin{frame}
\frametitle{References}
\footnotesize{
\begin{thebibliography}{99} % Beamer does not support BibTeX so references must be inserted manually as below
\bibitem[Smith, 2012]{p1} John Smith (2012)
\newblock Title of the publication
\newblock \emph{Journal Name} 12(3), 45 -- 678.
\end{thebibliography}
}
\end{frame}

%------------------------------------------------

\begin{frame}
\Huge{\centerline{The End}}
\end{frame}

%----------------------------------------------------------------------------------------

\end{document}